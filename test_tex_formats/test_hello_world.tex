\documentclass[a4paper,11pt]{article}
\usepackage {IEEEtrantools}
%Preamble to load other envirnments from some ./subfile.tex
\begin{document}
% subfile.tex
\usepackage{IEEEtrantools}
\LoadPackage{IEEEeqnarray}
\pspicture(1,1)
\psframe(1,1)
\endpspicture
\end{document}

% define the title
\author{H.~Partl}
\title{Minimalism}
\begin{document}
% generates the title
\maketitle
% insert the table of contents
\tableofcontents
\section{Some Interesting Words}
Well, and here begins my lovely article. \\

% Example 1
\ldots when Einstein introduced his formula
\begin{equation}
e = m \cdot c^2 \; ,
\end{equation}
which is at the same time the most widely known
and the least well understood physical formula.
% Example 2
\ldots from which follows Kirchhoff's current law:
\begin{equation}
\sum_{k=1}^{n} I_k = 0 \; .
\end{equation}
Kirchhoff's voltage law can be derived \ldots
% Example 3
\ldots which has several advantages.
\begin{equation}
I_D = I_F - I_R
\end{equation}
is the core of a very different transistor model. \ldots
\newpage

\TeX{} is pronounced as
$\tau\epsilon\chi$\\[5pt]
100~m$^{3}$ of water\\[5pt]
This comes from my $\heartsuit$ \\

This is text style:
$\lim_{n \to \infty}
\sum_{k=1}^n \frac{1}{k^2}
= \frac{\pi^2}{6}$.
And this is display style:

\begin{equation}
\lim_{n \to \infty}
\sum_{k=1}^n \frac{1}{k^2}
= \frac{\pi^2}{6} \\
\end{equation}

$\forall x \in \mathbf{R}:
\qquad x^{2} \geq 0$ \\

this works, but will this:\\
$\forall x \in \mathbf{R}:
\qquad x^{2} \geq 0$\\
$f(x) = x^2 \qquad f'(x)
= 2x \qquad f''(x) = 2\\[5pt]
\hat{XY} \quad \widehat{XY}
\quad \bar{x_0} \quad \bar{x}_0$\\

It worked! and now matrices:

\begin{equation*}
\mathbf{X} = \left(
\begin{array}{ccc}
x_1 & x_2 & \ldots \\
x_3 & x_4 & \ldots \\
\vdots & \vdots & \ddots
\end{array} \right)
\end{equation*}









\section{Good Bye World}
\ldots{} and here it ends.
\end{document}